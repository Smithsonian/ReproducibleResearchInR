% Options for packages loaded elsewhere
\PassOptionsToPackage{unicode}{hyperref}
\PassOptionsToPackage{hyphens}{url}
%
\documentclass[
]{article}
\usepackage{amsmath,amssymb}
\usepackage{iftex}
\ifPDFTeX
  \usepackage[T1]{fontenc}
  \usepackage[utf8]{inputenc}
  \usepackage{textcomp} % provide euro and other symbols
\else % if luatex or xetex
  \usepackage{unicode-math} % this also loads fontspec
  \defaultfontfeatures{Scale=MatchLowercase}
  \defaultfontfeatures[\rmfamily]{Ligatures=TeX,Scale=1}
\fi
\usepackage{lmodern}
\ifPDFTeX\else
  % xetex/luatex font selection
\fi
% Use upquote if available, for straight quotes in verbatim environments
\IfFileExists{upquote.sty}{\usepackage{upquote}}{}
\IfFileExists{microtype.sty}{% use microtype if available
  \usepackage[]{microtype}
  \UseMicrotypeSet[protrusion]{basicmath} % disable protrusion for tt fonts
}{}
\makeatletter
\@ifundefined{KOMAClassName}{% if non-KOMA class
  \IfFileExists{parskip.sty}{%
    \usepackage{parskip}
  }{% else
    \setlength{\parindent}{0pt}
    \setlength{\parskip}{6pt plus 2pt minus 1pt}}
}{% if KOMA class
  \KOMAoptions{parskip=half}}
\makeatother
\usepackage{xcolor}
\usepackage[margin=1in]{geometry}
\usepackage{color}
\usepackage{fancyvrb}
\newcommand{\VerbBar}{|}
\newcommand{\VERB}{\Verb[commandchars=\\\{\}]}
\DefineVerbatimEnvironment{Highlighting}{Verbatim}{commandchars=\\\{\}}
% Add ',fontsize=\small' for more characters per line
\usepackage{framed}
\definecolor{shadecolor}{RGB}{248,248,248}
\newenvironment{Shaded}{\begin{snugshade}}{\end{snugshade}}
\newcommand{\AlertTok}[1]{\textcolor[rgb]{0.94,0.16,0.16}{#1}}
\newcommand{\AnnotationTok}[1]{\textcolor[rgb]{0.56,0.35,0.01}{\textbf{\textit{#1}}}}
\newcommand{\AttributeTok}[1]{\textcolor[rgb]{0.13,0.29,0.53}{#1}}
\newcommand{\BaseNTok}[1]{\textcolor[rgb]{0.00,0.00,0.81}{#1}}
\newcommand{\BuiltInTok}[1]{#1}
\newcommand{\CharTok}[1]{\textcolor[rgb]{0.31,0.60,0.02}{#1}}
\newcommand{\CommentTok}[1]{\textcolor[rgb]{0.56,0.35,0.01}{\textit{#1}}}
\newcommand{\CommentVarTok}[1]{\textcolor[rgb]{0.56,0.35,0.01}{\textbf{\textit{#1}}}}
\newcommand{\ConstantTok}[1]{\textcolor[rgb]{0.56,0.35,0.01}{#1}}
\newcommand{\ControlFlowTok}[1]{\textcolor[rgb]{0.13,0.29,0.53}{\textbf{#1}}}
\newcommand{\DataTypeTok}[1]{\textcolor[rgb]{0.13,0.29,0.53}{#1}}
\newcommand{\DecValTok}[1]{\textcolor[rgb]{0.00,0.00,0.81}{#1}}
\newcommand{\DocumentationTok}[1]{\textcolor[rgb]{0.56,0.35,0.01}{\textbf{\textit{#1}}}}
\newcommand{\ErrorTok}[1]{\textcolor[rgb]{0.64,0.00,0.00}{\textbf{#1}}}
\newcommand{\ExtensionTok}[1]{#1}
\newcommand{\FloatTok}[1]{\textcolor[rgb]{0.00,0.00,0.81}{#1}}
\newcommand{\FunctionTok}[1]{\textcolor[rgb]{0.13,0.29,0.53}{\textbf{#1}}}
\newcommand{\ImportTok}[1]{#1}
\newcommand{\InformationTok}[1]{\textcolor[rgb]{0.56,0.35,0.01}{\textbf{\textit{#1}}}}
\newcommand{\KeywordTok}[1]{\textcolor[rgb]{0.13,0.29,0.53}{\textbf{#1}}}
\newcommand{\NormalTok}[1]{#1}
\newcommand{\OperatorTok}[1]{\textcolor[rgb]{0.81,0.36,0.00}{\textbf{#1}}}
\newcommand{\OtherTok}[1]{\textcolor[rgb]{0.56,0.35,0.01}{#1}}
\newcommand{\PreprocessorTok}[1]{\textcolor[rgb]{0.56,0.35,0.01}{\textit{#1}}}
\newcommand{\RegionMarkerTok}[1]{#1}
\newcommand{\SpecialCharTok}[1]{\textcolor[rgb]{0.81,0.36,0.00}{\textbf{#1}}}
\newcommand{\SpecialStringTok}[1]{\textcolor[rgb]{0.31,0.60,0.02}{#1}}
\newcommand{\StringTok}[1]{\textcolor[rgb]{0.31,0.60,0.02}{#1}}
\newcommand{\VariableTok}[1]{\textcolor[rgb]{0.00,0.00,0.00}{#1}}
\newcommand{\VerbatimStringTok}[1]{\textcolor[rgb]{0.31,0.60,0.02}{#1}}
\newcommand{\WarningTok}[1]{\textcolor[rgb]{0.56,0.35,0.01}{\textbf{\textit{#1}}}}
\usepackage{graphicx}
\makeatletter
\newsavebox\pandoc@box
\newcommand*\pandocbounded[1]{% scales image to fit in text height/width
  \sbox\pandoc@box{#1}%
  \Gscale@div\@tempa{\textheight}{\dimexpr\ht\pandoc@box+\dp\pandoc@box\relax}%
  \Gscale@div\@tempb{\linewidth}{\wd\pandoc@box}%
  \ifdim\@tempb\p@<\@tempa\p@\let\@tempa\@tempb\fi% select the smaller of both
  \ifdim\@tempa\p@<\p@\scalebox{\@tempa}{\usebox\pandoc@box}%
  \else\usebox{\pandoc@box}%
  \fi%
}
% Set default figure placement to htbp
\def\fps@figure{htbp}
\makeatother
\setlength{\emergencystretch}{3em} % prevent overfull lines
\providecommand{\tightlist}{%
  \setlength{\itemsep}{0pt}\setlength{\parskip}{0pt}}
\setcounter{secnumdepth}{-\maxdimen} % remove section numbering
\usepackage{bookmark}
\IfFileExists{xurl.sty}{\usepackage{xurl}}{} % add URL line breaks if available
\urlstyle{same}
\hypersetup{
  pdftitle={Module 1: Introduction to R},
  pdfauthor={Brian P Steves},
  hidelinks,
  pdfcreator={LaTeX via pandoc}}

\title{Module 1: Introduction to R}
\author{Brian P Steves}
\date{2024-09-27}

\begin{document}
\maketitle

\section{Module 1: Introduction to R}\label{module-1-introduction-to-r}

\subsection{1 Installing R}\label{installing-r}

R is open source, free, and installs on most computing platforms
including Linux, Windows and MacOS. You should be able to find a
compatible download of the most recent version of R at the follwing URL.

\url{http://www.r-project.org/}

Follow the instructions provided for you operating system.

\subsection{2 Installing RStudio IDE (inegrated development
environment)}\label{installing-rstudio-ide-inegrated-development-environment}

RStudio an integrated development environment for R meaning it provides
a suite of tools to help you interact with R.

It contains a code editor with sntax highlighting, code completion and
smart indentation. It also provides a well thought out layout for you to
access your command history, current list of objects, plots, help,
files, packages, etc.. from within on application.

There are even some more advanced features concerning report generation
and version control integration which we will cover towards the end of
class.

RStudio is also free, available for Windows, MacOS, and linux and can be
downloaded for installation from the following URL.

\url{http://www.rstudio.com/}

Follow the instructions provided for you operating system.

\subsection{3 Basics of R}\label{basics-of-r}

\subsubsection{3.1 Interactive R on the
Console}\label{interactive-r-on-the-console}

In other words, entering code line by line rather than from a script.

\paragraph{3.1.1 Starting and stopping R}\label{starting-and-stopping-r}

To start R from the command line simply type ``R'' and you'll get
something like the following.

\begin{verbatim}
R version 3.0.3 (2014-03-06) -- "Warm Puppy"
Copyright (C) 2014 The R Foundation for Statistical Computing
Platform: x86_64-pc-linux-gnu (64-bit)

R is free software and comes with ABSOLUTELY NO WARRANTY.
You are welcome to redistribute it under certain conditions.
Type 'license()' or 'licence()' for distribution details.

  Natural language support but running in an English locale

R is a collaborative project with many contributors.
Type 'contributors()' for more information and
'citation()' on how to cite R or R packages in publications.

Type 'demo()' for some demos, 'help()' for on-line help, or
'help.start()' for an HTML browser interface to help.
Type 'q()' to quit R.

>
\end{verbatim}

In RStudio, this console has its own window called ``Console'' and
should start up automatically when you start RStudio.

To quit R use the following command at the console prompt

\begin{Shaded}
\begin{Highlighting}[]
\FunctionTok{q}\NormalTok{()}
\end{Highlighting}
\end{Shaded}

of if you quit RStudio it will quit you out of R.

\paragraph{3.1.2 Getting help in R}\label{getting-help-in-r}

You can get help on a specific topic using the ``help()'' function. An
alternative shortcut to help is ``?''

\begin{Shaded}
\begin{Highlighting}[]
\FunctionTok{help}\NormalTok{(}\StringTok{"lm"}\NormalTok{)}
\end{Highlighting}
\end{Shaded}

While using ``?'' is usually easier, there are time when the full help
function is better. For example ``?if'' will not evaluate while
``help(`if')'' returns a help file about ``Control Flow'' which is
probably what you want.

You can also search the help files using ``help.search()'' function when
you don't know the exact command you want help on. The short cut to this
is to use ``??''.

\begin{Shaded}
\begin{Highlighting}[]
\FunctionTok{help.search}\NormalTok{(}\StringTok{"lm"}\NormalTok{)}
\end{Highlighting}
\end{Shaded}

To run examples from a paticular function\ldots{}

\begin{Shaded}
\begin{Highlighting}[]
\FunctionTok{example}\NormalTok{(}\StringTok{"lm"}\NormalTok{)}
\end{Highlighting}
\end{Shaded}

\paragraph{3.1.3 Assignment of data structures to
objects}\label{assignment-of-data-structures-to-objects}

To can assign a data structure to an object using ``\textless-'' or
``-\textgreater{}'' or the ``assign()'' function. The following are all
equivalent.

\begin{Shaded}
\begin{Highlighting}[]
\NormalTok{x}\OtherTok{\textless{}{-}}\FunctionTok{c}\NormalTok{(}\DecValTok{10}\NormalTok{,}\DecValTok{7}\NormalTok{,}\DecValTok{4}\NormalTok{,}\DecValTok{5}\NormalTok{)}

\FunctionTok{c}\NormalTok{(}\DecValTok{10}\NormalTok{,}\DecValTok{7}\NormalTok{,}\DecValTok{4}\NormalTok{,}\DecValTok{5}\NormalTok{) }\OtherTok{{-}\textgreater{}}\NormalTok{ x}

\FunctionTok{assign}\NormalTok{(}\StringTok{"x"}\NormalTok{, }\FunctionTok{c}\NormalTok{(}\DecValTok{10}\NormalTok{,}\DecValTok{7}\NormalTok{,}\DecValTok{4}\NormalTok{,}\DecValTok{5}\NormalTok{))}
\end{Highlighting}
\end{Shaded}

Calling the object by name with print it

\begin{Shaded}
\begin{Highlighting}[]
\NormalTok{x}
\end{Highlighting}
\end{Shaded}

\begin{verbatim}
## [1] 10  7  4  5
\end{verbatim}

If you don't assign a result to an object the resulting value is printed
and lost.

\begin{Shaded}
\begin{Highlighting}[]
\FunctionTok{c}\NormalTok{(}\DecValTok{10}\NormalTok{,}\DecValTok{7}\NormalTok{,}\DecValTok{4}\NormalTok{,}\DecValTok{5}\NormalTok{)}
\end{Highlighting}
\end{Shaded}

\begin{verbatim}
## [1] 10  7  4  5
\end{verbatim}

\paragraph{3.1.4 Listing of Objects}\label{listing-of-objects}

All of the current ojects in memory can be listed using the ``ls()'' or
``objects()'' functions.

\begin{Shaded}
\begin{Highlighting}[]
\FunctionTok{ls}\NormalTok{()}
\end{Highlighting}
\end{Shaded}

\begin{verbatim}
## [1] "x"
\end{verbatim}

\paragraph{3.1.5 Recalling previous
commands}\label{recalling-previous-commands}

While at the console, the up and down arrows will cycle the command
prompt with the previously executed commands.

\subsubsection{3.2 R Script files}\label{r-script-files}

Entering lines of code into the console might be fun but we can combine
a series of commands into an R script file. These files are simply text
files that end with the ``.r'' or ``.R'' extension and contain a series
of R commands.

Once an R script is written and saved it can be evaluated in R using the
``source()'' command.

If you plan on running more than a couple lines of code it really helps
to use script files as it is much easier to edit a script file and rerun
it than it is to retype a long series of commands over again.

\begin{Shaded}
\begin{Highlighting}[]
\FunctionTok{source}\NormalTok{(}\StringTok{"myScript.R"}\NormalTok{)}
\end{Highlighting}
\end{Shaded}

\subsection{4 Contributed Packages}\label{contributed-packages}

One of the great things about R is the diversity of code that has been
written and is freely available to everyone. The most common way people
share R code is as a package of functions, usually with a unifing theme.

\subsubsection{4.1 Finding Packages}\label{finding-packages}

\paragraph{4.1.1 CRAN Task Views}\label{cran-task-views}

Finding an appropriate package can be a bit of hit or miss as the
quality ranges from exceptional to awful. One good place to start are
the ``Task Views'' maintained on the main R project server.

\url{http://cran.r-project.org/web/views/}

For example, if you are interested in ``Reproducible Research'' there is
a Task View for that\ldots{}
\url{http://cran.r-project.org/web/views/ReproducibleResearch.html}

This task view is an annotated list of packages that relate to
reproducible research.

Another important task view for ecologists and environmental scientists
is the ``Analysis of Ecological and Environmental Data'' task
view\ldots{}

\url{http://cran.r-project.org/web/views/Environmetrics.html}

\paragraph{4.1.2 Searching Google}\label{searching-google}

If you already know which statistical technique you are looking to
employ but don't know which R package might need you can always search
Google for it. Unfortunately, the letter ``R'' isn't much of a unique
name for a Google search on R code. However, if you change ``R'' to
``CRAN R'' in that query to refer to the ``Comprehensive R Archive
Network'' you will often get better results.

\subsubsection{4.2 Installing Packages}\label{installing-packages}

Once you know which R package you want to install you use the
``install.packages()'' command to install it.

\begin{Shaded}
\begin{Highlighting}[]
\FunctionTok{install.packages}\NormalTok{(}\StringTok{"vegan"}\NormalTok{, }\AttributeTok{repos=}\StringTok{\textquotesingle{}http://cran.us.r{-}project.org\textquotesingle{}}\NormalTok{)}
\end{Highlighting}
\end{Shaded}

\begin{verbatim}
## 
##   There is a binary version available but the source version is later:
##       binary source needs_compilation
## vegan  2.6-4  2.6-8              TRUE
\end{verbatim}

If you haven't told R or RStudio where you want to download this package
from, it may ask you for you to select a mirror. You'll probably want to
pick either the closest mirror server or one that you feel is the
quickest. Once the package is downloaded, R will attempt to install it
on your computer. If anything goes wrong in the installation an error
will be reported. Packages can be installed for all users on the
computer or just one specific user. If you don't have full admin rights
on your computer your package will be installed in a local user library
of packages.

\subsubsection{4.3 Loading Packages}\label{loading-packages}

To load an install package for use you use the ``library()'' function.

\begin{Shaded}
\begin{Highlighting}[]
\FunctionTok{library}\NormalTok{(vegan)}
\end{Highlighting}
\end{Shaded}

\subsection{Homework}\label{homework}

\begin{enumerate}
\def\labelenumi{\arabic{enumi}.}
\tightlist
\item
  Install the latest versions of R onto your computer.
\item
  Install the latest version of RStudio onto your computer.
\item
  Find and install at least one R package.
\end{enumerate}

\end{document}
